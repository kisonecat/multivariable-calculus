\documentclass{ximera}

\title{Geometric series}

\begin{document}

\begin{abstract}
  Geometric series are a useful example to consider when thinking about power series.
\end{abstract}

\maketitle

Power series are useful for more than just \(e^x\).  Starting from a series that we know, like \(\frac{1}{1-x} = \sum_{n=0}^\infty x^n\) for \(|x| < 1\), we can integrate and differentiate to build power series for other functions.

\youtube{https://www.youtube.com/watch?v=jn_5dL1MJYI}

\begin{question}
  Suppose \(\displaystyle\sum_{n=1}^\infty b_n = \displaystyle\frac{16 \, x^{3} }{ {\left(16 \, x^{4} - 1\right)}^{2} }\).  Which of the following could be an expression for \(b_n\)?  Note that \(b_n\) will involve \(x\).
  
  \begin{solution}
    \begin{hint}
      Let's call \(F(x) = \displaystyle\frac{16 \, x^{3} }{ {\left(16 \, x^{4} - 1\right)}^{2} }\).
    \end{hint}
    \begin{hint}
      Note that \(F(x) = \displaystyle\frac{d}{dx} \left( -\displaystyle\frac{1}{4 \, {\left(16 \, x^{4} - 1\right)} } \right)\).
    \end{hint}
    \begin{hint}
      Let's call \(f(x) = -\displaystyle\frac{1}{4 \, {\left(16 \, x^{4} - 1\right)} }\) so that \(F(x) = f'(x)\).
    \end{hint}
    \begin{hint}
      Note that \(\displaystyle\frac{1}{1 - \left(16 \, x^{4}\right)} = \displaystyle\sum_{n=0}^\infty \left( 16 \, x^{4} \right)^n \).
    \end{hint}
    \begin{hint}
      Consequently \(f(x) = \displaystyle\frac{1}{4 \left( 1 - \left(16 \, x^{4}\right)\right)} = \displaystyle\sum_{n=0}^\infty \displaystyle\frac{\left( 16 \, x^{4} \right)^n}{4} \).
    \end{hint}
    \begin{hint}
      Therefore \(F(x) = f'(x) = \displaystyle\frac{d}{dx} \displaystyle\sum_{n=0}^\infty \displaystyle\frac{\left( 16 \, x^{4} \right)^n}{4} \).
    \end{hint}
    \begin{hint}
      We could write this as \(F(x) = \displaystyle\frac{d}{dx} \displaystyle\sum_{n=0}^\infty \displaystyle\frac{\left( 2^{4 \, n} x^{4 \, n} \right)}{4} \).
    \end{hint}
    \begin{hint}
      But power series can be differentiated term-by-term.
    \end{hint}
    \begin{hint}
      Therefore, \(F(x) = \displaystyle\sum_{n=1}^\infty \displaystyle\frac{d}{dx} \displaystyle\frac{\left( 2^{4 \, n} x^{4 \, n} \right)}{4} \).
    \end{hint}
    \begin{hint}
      And so \(F(x) = \displaystyle\sum_{n=1}^\infty \left( 2^{4 \, n} n x^{4 \, n - 1} \right) \).
    \end{hint}
    \begin{hint}
      We conclude that \(b_n = 2^{4 \, n} n x^{4 \, n - 1} \).
    \end{hint}


    \begin{multiple-choice}
      \choice[correct]{\(2^{4 \, n} n x^{4 \, n - 1}\)}
      \choice{\(2^{3 \, n} n x^{3 \, n - 1}\)}
      \choice{\(4^{2 \, n} n x^{2 \, n - 1}\)}
      \choice{\(3^{3 \, n} n x^{3 \, n - 1}\)}
      \choice{\(3^{2 \, n} n x^{2 \, n - 1}\)}
    \end{multiple-choice}

  \end{solution}
\end{question}

\hrule

\section{What is a formula for the Fibonacci numbers?}

Thinking about power series brings us naturally into the much more advanced topic of \href{http://en.wikipedia.org/wiki/Generating_function}{generating functions}.

Let \((a_n)\) be the sequence of \href{http://en.wikipedia.org/wiki/Fibonacci_number}{Fibonacci numbers}, and consider \(f(x) = \sum_{n=0}^\infty a_n x^n\), the power series with coefficients equal to the Fibonacci numbers.  By messing around with this power series, we will be able to write down a formula for \(a_n\) in terms of \(n\) and, surprisingly, \(\sqrt{5}\).  The \href{http://en.wikipedia.org/wiki/Golden_ratio}{Golden ratio} makes a cameo, too.

\youtube{https://www.youtube.com/watch?v=CR-nmp97Ayo}


\hrule

\section{Putting it all together}

\begin{question}
  Consider the function \[f(t) = \displaystyle\int_0^t e^{-x^2} \, dx.\]  This is an an important function in probability, but it can be hard to work with because it isn't possible to write down an antiderivative for \(e^{-x^2}\) using the ``elementary'' functions we have on hand.  So if we want to compute \(f(\displaystyle\frac{3}{2})\), we'll have to do some work.  Put together what you know about \(e^x\), about integrating power series, and about approximating alternating series to evaluate \(f(\displaystyle\frac{3}{2})\) to within \(\displaystyle\frac{1}{2}\).

I imagine you may want to use a computer to do some arithmetic, but using the ``erf'' button might spoil the fun---assuming you have one, and you rescale it appropriately.

\begin{solution}
  \begin{hint}
    We would like to find a series for \(f(t) = \displaystyle\int_0^t e^{-x^2} \, dx\).
  \end{hint}
  \begin{hint}
    One helpful thing is that \(e^x = \displaystyle\sum_{n=0}^\infty \displaystyle\frac{x^n}{n!}\).
  \end{hint}
  \begin{hint}
    This is a tricky step: you might then believe that \(e^{-x^2} = \displaystyle\sum_{n=0}^\infty \displaystyle\frac{\left(-x^2\right)^n}{n!}\).
  \end{hint}
  \begin{hint}
    So we can simplify this to get \(e^{-x^2} = \displaystyle\sum_{n=0}^\infty \displaystyle\frac{(-1)^n \, x^{2n} }{n!}\).
  \end{hint}
  \begin{hint}
    That looks a bit like an alternating series!  That is a good sign (!).
  \end{hint}
  \begin{hint}
    Another thing we can do is integrate a power series term-by-term.
  \end{hint}
  \begin{hint}
    That means that \(\displaystyle\int_{x=0}^t e^{-x^2} \, dx = \displaystyle\int_{x=0}^t \displaystyle\sum_{n=0}^\infty \displaystyle\frac{(-1)^n \, x^{2n} }{n!} dx = \displaystyle\sum_{n=0}^\infty \displaystyle\int_{x=0}^t \displaystyle\frac{(-1)^n x^{2n} }{n!} dx\).
  \end{hint}
  \begin{hint}
    But \(\displaystyle\int_{x=0}^t x^{2n} \, dx = \displaystyle\frac{x^{2n+1} }{2n+1}\).
  \end{hint}
  \begin{hint}
    So \(f(t) = \displaystyle\int_{x=0}^t e^{-x^2} \, dx = \displaystyle\sum_{n=0}^\infty \displaystyle\frac{(-1)^n x^{2n+1} }{n! \cdot (2n+1)}\).
  \end{hint}
  \begin{hint}
    We want to approximate \(f\left(\displaystyle\frac{3}{2}\right) = \displaystyle\sum_{n=0}^\infty \displaystyle\frac{(-1)^n \left(\displaystyle\frac{3}{2}\right)^{2n+1} }{n! \cdot (2n+1)}\).
  \end{hint}
  \begin{hint}
    This is an alternating series, so we know how to approximate the error in a partial sum.
  \end{hint}
  \begin{hint}
    We look for a term in the series with absolute value no larger than \(\displaystyle\frac{1}{2}\).
  \end{hint}
  \begin{hint}
    So we can talk about this more easily, set \(a_{n} = \displaystyle\frac{\left(\displaystyle\frac{3}{2}\right)^{2 \, n + 1} \left(-1\right)^{n} }{ {\left(2 \, n + 1\right)} n!}\).
  \end{hint}
  \begin{hint}
    Plugging in a few values, we find that when \(n = 3\), then \(|a_{3}| = \displaystyle\frac{729}{1792}\) which is less than \(\displaystyle\frac{1}{2}\).
  \end{hint}
  \begin{hint}
    So the true value of the series is between \(\displaystyle\sum_{n = 0}^{2} \displaystyle\frac{\left(\displaystyle\frac{3}{2}\right)^{2 \, n + 1} \left(-1\right)^{n} }{ {\left(2 \, n + 1\right)} n!}\) and \(\displaystyle\sum_{n = 0}^{3} \displaystyle\frac{\left(\displaystyle\frac{3}{2}\right)^{2 \, n + 1} \left(-1\right)^{n} }{ {\left(2 \, n + 1\right)} n!}\).
  \end{hint}
  \begin{hint}
    Doing some arithmetic, \(\displaystyle\sum_{n = 0}^{2} \displaystyle\frac{\left(\displaystyle\frac{3}{2}\right)^{2 \, n + 1} \left(-1\right)^{n} }{ {\left(2 \, n + 1\right)} n!} = \displaystyle\frac{363}{320}\).
  \end{hint}
  \begin{hint}
    Doing some more arithmetic, \(\displaystyle\sum_{n = 0}^{3} \displaystyle\frac{\left(\displaystyle\frac{3}{2}\right)^{2 \, n + 1} \left(-1\right)^{n} }{ {\left(2 \, n + 1\right)} n!} = \displaystyle\frac{6519}{8960}\).
  \end{hint}
  \begin{hint}
    Looking over the possible choices, \(1\) is between \(\displaystyle\frac{363}{320}\) and \(\displaystyle\frac{6519}{8960}\).
  \end{hint}
  \begin{hint}
    So we conclude that \(\left| L - 1 \right| < \displaystyle\frac{1}{2}\).
  \end{hint}
  \begin{hint}
    In other words, \(L\) is approximately \(1\) with an error of no more than \(\displaystyle\frac{1}{2}\).
    
  \end{hint}
  
  \begin{multiple-choice}
    \choice[correct]{\(1\)}
    \choice{\(\displaystyle\frac{3}{2}\)}
    \choice{\(2\)}
    \choice{\(3\)}
  \end{multiple-choice}

\end{solution}
\end{question}
            


\end{document}
