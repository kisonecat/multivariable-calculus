\documentclass{ximera}

\title{Radius of convergence}

\begin{document}

\begin{abstract}
  How big is the interval on which the power series converges?

  Power series will lead the way to Taylor series.
\end{abstract}

\maketitle

For a power series of the form \(\sum_{n=0}^\infty a_n x^n\), the interval of convergence necessarily looks like \((-R,R)\) or \([-R,R]\) or maybe a half-open interval like \((-R,R]\) or \([-R,R)\).  That number, \(R\), is called the radius of convergence.

\youtube{https://www.youtube.com/watch?v=ioNDxuLes3g}

It is possible to say that \(R = \infty\), meaning that the power series converges everywhere.

\youtube{https://www.youtube.com/watch?v=Vo_arekknd0}

On the other hand, it is also possible to find a situation where \(R = 0\), meaning that the power series converges only at the origin.

\youtube{https://www.youtube.com/watch?v=dCBh-f9CMRw}

% Relevant video: radius-of-convergence
\begin{question}
  What is the radius of convergence of the series \(\displaystyle\displaystyle\sum_{n=3}^\infty \left( \displaystyle\frac{3 \, n x^{n}}{2^{n} + 3 \, n} \right)\)?
  
  \begin{solution}
    \begin{hint}
      Let \(a_{n} = \displaystyle\frac{3 \, n}{2^{n} + 3 \, n}\), so the given series is \(\displaystyle\sum_{n=3}^\infty a_{n} x^{n}\).
    \end{hint}
    \begin{hint}
      For the ratio test, consider \(\lim_{n \to \infty} \displaystyle\frac{|a_{n+1}|}{|a_{n}|}\).
    \end{hint}
    \begin{hint}
      We want this limit to be less than one to get absolute convergence.
    \end{hint}
    \begin{hint}
      In this case, \(\lim_{n \to \infty} \displaystyle\frac{|a_{n+1} x^{n+1}|}{|a_{n} x^{n}|} = \lim_{n \to \infty} \left| \displaystyle\frac{\displaystyle\frac{3 \, {\left(n + 1\right)}}{2^{n + 1} + 3 \, n + 3}}{\displaystyle\frac{3 \, n}{2^{n} + 3 \, n}} \right| \cdot \left| x \right|\).
    \end{hint}
    \begin{hint}
      Simplifying, \(\lim_{n \to \infty} \displaystyle\frac{|a_{n+1} x^{n+1}|}{|a_{n} x^{n}|} = \lim_{n \to \infty} \left| \displaystyle\frac{{\left(2^{n} + 3 \, n\right)} {\left(n + 1\right)}}{{\left(2^{n + 1} + 3 \, n + 3\right)} n} \right| \cdot \left| x \right|\).
    \end{hint}
    \begin{hint}
      Multiply the numerator and denominator by \(1/2^n\).
    \end{hint}
    \begin{hint}
      Then \(\lim_{n \to \infty} \displaystyle\frac{|a_{n+1} x^{n+1}|}{|a_{n} x^{n}|} = \lim_{n \to \infty} \left| 
        \displaystyle\frac{n + \displaystyle\frac{3 \, n^{2}}{2^{n}} + \displaystyle\frac{3 \, n}{2^{n}} + 1}{2 \, n + \displaystyle\frac{3 \, n^{2}}{2^{n}} + \displaystyle\frac{3 \, n}{2^{n}}} \right| \cdot \left| x \right|\).
    \end{hint}
    \begin{hint}
      So \(\lim_{n \to \infty} \displaystyle\frac{|a_{n+1} x^{n+1}|}{|a_{n} x^{n}|} = \displaystyle\frac{1}{2} \cdot \left| x \right|\).
    \end{hint}
    \begin{hint}
      Therefore the limit is less than one when \(\left| x \right| < 2\).
    \end{hint}
    \begin{hint}
      So the radius of convergence is \(2\).
    \end{hint}

    \begin{multiple-choice}
      \choice[correct]{\(2\)}
      \choice{\(\displaystyle\frac{1}{2}\)}
      \choice{\(5\)}
      \choice{\(\displaystyle\frac{1}{6}\)}
      \choice{\(\displaystyle\frac{1}{3}\)}
    \end{multiple-choice}

  \end{solution}
\end{question}
            
\hrule

\section{What if I'd like a power series in terms of \((x-c)\)?}

So far, we have been considering power series of the form \(\sum_{n=0}^\infty a_n x^n\), but it can be helpful to consider the case \(\sum_{n=0}^\infty a_n (x-c)^n\), which we think of as a power series centered around a different point.

\youtube{https://www.youtube.com/watch?v=Cz0OWRWRAjw}


\end{document}
