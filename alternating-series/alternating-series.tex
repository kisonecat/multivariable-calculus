\documentclass{ximera}

\title{Alternating series}

\begin{document}

\begin{abstract}
  Series with some positive and some negative terms can be difficult to analyze.  If the signs flip-flop, then the alternating series test provides a nice criterion.
\end{abstract}

\maketitle

We now consider absolute and conditional convergence, alternating
series and the alternating series test.

Previously, we considered convergence tests; now we consider
convergence for series with some negative and some positive terms.  Up
until now, we had been considering series with nonnegative terms; it
is much easier to determine convergence when the terms are
nonnegative!

\youtube{https://www.youtube.com/watch?v=M058Piz2Ic4}

Now that we will consider series with both negative and positive
terms, there will definitely be some new complications.

\begin{remark}
  And in a certain sense, this is the end of ``Does it converge?'' as
  a theme for our course.  From here on out, we consider power series
  and then, Taylor series.  Those last two topics will move us away
  from questions of mere convergence, so if you have been eager for
  new material, stay tuned.!
\end{remark}

% <p>As always, I have some learning outcomes for you to strive for.  At the end of this week, I hope you will be able to</p>
% <ul>
% <li>relate <%= linkto_video('absolute-convergence','absolute convergence to convergence') %>,</li>
% <li>apply the <%= linkto_video('limit-comparison-test','limit comparison test') %> to analyze convergence,</li>
% <li>determine whether an <%= linkto_video('alternating-series','alternating \\(p\\)-series converges') %>,</li>
% <li>apply the <%= linkto_video('alternating-series-test','alternating series test') %> when appropriate,</li>
% <li>find an <%= linkto_video('alternating-series-important','approximate value')%> for an alternating series,</li>
% <li>approximate the value of a <%= linkto_video('e-is-irrational','natural logarithm') %>, and finally</li>
% <li>consider the alternating series test in a <%= linkto_video('monotone-important-for-ast','nonmonotone situation') %>.</li>
% </ul>
% <p>As soon as you have time, you can get started <a href="/sequence-001/quiz/attempt?quiz_id=151"><i class="icon-pencil"></i>&nbsp;thinking about the homework</a>.  The homework is &ldquo;formative assessment&rdquo; meaning I intend the homework as a teaching tool&mdash;and not to determine your success as a mathematician.  <em>Greatness consists not in being great, but in becoming greater,</em> so I hope you'll use the resources in this course to help you learn more than you knew about series before.  Use the hints.  Discuss with others on the forum.  With your hard work, you can master these concepts.</p>

\section{How is convergence affected by the choice of initial index?}

Not at all!  For example, \(\sum_{n=1}^\infty a_n\) converges precisely when \(\sum_{n=17}^\infty a_n\) converges---whether the starting index is \(n=1\) or \(n = 17\) or \(n = 1000\), it doesn't matter.

In short, convergence only depends on the tail.

\youtube{https://www.youtube.com/watch?v=wNFnB9kN7_M}


\end{document}
