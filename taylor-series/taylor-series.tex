\documentclass{ximera}

\title{Taylor series}

\begin{document}

\begin{abstract}
  Many popular functions can be written as power series.
\end{abstract}

\maketitle

Previously, we met power series.  The big change for us was that,
instead of considering a single series, last week we started
considering series like \(\sum_{n=0}^\infty a_n x^n\) that included a
parameter \(x\).  We learned how to recognize some power series as
well-known functions for certain values of \(x\), like
\(\sum_{n=0}^\infty x^n = \frac{1}{1-x}\) when \(|x| < 1\).  You are
perhaps wondering what we will be studying next\ldots It's time for
Taylor series.  Instead of starting with a power series and finding a
nice description of the function it represents, we will start with a
function, and try to find a power series for it.  There is no
guarantee of success!  But incredibly, many of our favorite functions
will have power series representations.  \textbf{Sometimes dreams come
  true!}

\end{document}
