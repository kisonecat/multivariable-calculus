\documentclass{ximera}

\title{Sequences}

\begin{document}

\begin{abstract}
  A sequence of numbers is an unending list of numbers.  Sequences form the foundation for our later discussion of series.
\end{abstract}

\maketitle

First off, welcome to the course!  My name is \href{http://kisonecat.com/}{Jim Fowler}, and I am very glad that you are here.

\textbf{This course includes ``prelectures.''}  Today's the first
lecture, so this is a \textit{postlecture} and there is \textbf{a ton
  of material} here on sequences.  I posted quite a large amount of
material on sequences with the idea that, after we survey the material
in class today, you'll go home and look over the material.  When you
are reviewing what you know about sequences, I have some goals for you
to aspire to.  I'd like you to be able to
\begin{itemize}
\item compute terms of a sequence presented as a recursive formula,
\item define and compute limits of sequences,
\item find a sufficiently large index to guarantee a sequence is within \(\epsilon\) of \(L\),
\item define what it means for a sequence to be bounded and determine when a sequence is bounded,
\item identify an arithmetic progression,
\item identify a geometric progression,
\item identify a monotonic sequence, and
\item apply the monotone convergence theorem.
\end{itemize}
In stark contrast to this postlecture, most lectures will include a
\textbf{much shorter} prelecture assignment; you can get started on
Friday's prelecture by skipping ahead to the section called ``Series''
and watching the videos there and working the exercises.

You can find all the videos I prepared on sequences and series by going to \href{https://www.youtube.com/playlist?list=PLjOkVtsM_edJgB9yG6fRb5VXsoSxQN0L5}{my YouTube channel}.

\hrule

\section{Review Questions for Sequences}

\subsection{What are some examples?}

There are a ton of interesting sequences to consider; important examples include arithmetic progressions and geometric progressions.

\subsection{What is the limit of a sequence?}

If you have got a list of numbers, a natural question is whether that list of numbers is getting close to anything in particular; this is the idea behind the limit of a sequence.  The precise definition of limit---in terms of \(\epsilon\) and \(N\)---can seem very complicated.  It can help to think through some concrete examples where we can determine how large \(N\) needs to be for a given \(\epsilon\).

There are plenty of examples where this can be quite difficult to do.

\youtube{https://www.youtube.com/watch?v=X_v9Y3jLrfA}

\subsection{Why do we care about any of this?}

Much of what we are doing with sequences is setting up machinery that we'll make use of in the future.  This material is something like the ``review of functions'' material from Calculus One: we're just introducing terminology that we'll make use of later.  But already there are reasons to care: we can, for instance, use sequences to approximate \(\sqrt{2}\).

\youtube{https://www.youtube.com/watch?v=nV01XC-seVM}

Although this is a useful application, for me, the most interesting application of mathematics is to the human spirit.  It's just fun to think about!

\subsection{What other properties might a sequence have?}

A sequence may be \textbf{bounded}, meaning all its terms are above a ``lower bound'' and below an
``upper bound.''  A sequence may be increasing (or decreasing), meaning its terms are getting larger (or smaller) as we go out farther in the sequence.  
Another word, ``monotone,'' is used to speak about sequences which are heading in the same direction, so increasing sequences and decreasing sequences are both examples of monotone sequences.

This might not sound too important, but monotonicity and boundedness together imply converegence!  This remarkable theorem is  helpful because the theorem guarantees a limit exists, even when it might be hard to actually compute it.

\subsection{How big can sequences be?}

Although it is not, strictly speaking, one of the learning objectives of the course, it is interesting to think about how ``big'' a sequence can be.  For example, can you think of a sequence which lists every integer?

\youtube{https://www.youtube.com/watch?v=AyAHoCi9U68}

Can you think of a sequence which lists every real number?  That a collection of things can be listed off in a sequence is a more restrictive condition than it might initially appear.

\youtube{https://www.youtube.com/watch?v=GWGDkKfv7bs}

\end{document}
