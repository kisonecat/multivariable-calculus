\documentclass{ximera}

\title{Fibonacci numbers}

\newcommand{\defnword}[1]{\textbf{#1}}
\newcommand{\ds}{\displaystyle}
\newcommand{\Z}{\mathbb{Z}}
\newcommand{\N}{\mathbb{N}}
\newcommand{\nth}{\mbox{\scriptsize th}}
\renewcommand{\index}[1]{}

\begin{document}

\begin{abstract}
  The Fibonacci numbers have a nice recursive definition.
\end{abstract}

\maketitle

%\marginnote{The Fibonacci numbers are interesting enough that a
%  journal, \href{http://www.fq.math.ca/}{The Fibonacci Quarterly} is
%  published four times yearly entirely on topics related to the
%  Fibonacci numbers.}

The \defnword{Fibonacci numbers} are defined recursively, starting with
$$
F_0 = 0 \mbox{ and } F_1 = 1
$$
and the rule that $F_{n} = F_{n-1} + F_{n-2}$.  We can restate this
formula in words, instead of symbols; stated in words, each term is
the sum of the previous two terms.  So the sequence of Fibonacci
numbers begins 
$$
0, \quad 1, \quad 1, \quad 2, \quad 3, \quad 5, \quad 8, \quad 13, \quad 21, \quad 34, \quad\ldots
$$
and continues.

This is certainly not the last time we will see the Fibonacci numbers.

\end{document}
