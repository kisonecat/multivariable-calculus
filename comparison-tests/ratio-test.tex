\documentclass{ximera}

\title{Ratio test}

\begin{document}

\begin{abstract}
  The ratio test begins by considering the limit of the ratio of subsequent terms in a series.
\end{abstract}

\maketitle

\section{What is the ratio test?}

The geometric series \(\displaystyle\sum_{n=1}^\infty \displaystyle\frac{1}{4^n}\) converges because the common ratio between its terms is \(\displaystyle\frac{1}{4} < 1\); determining whether \(\displaystyle\sum_{n=1}^\infty \displaystyle\frac{n^5}{4^n}\) converges requires something more delicate.

\youtube{https://www.youtube.com/watch?v=l1akoXWmhSM}

The underlying technique is the ratio test.

\youtube{https://www.youtube.com/watch?v=2gi7pyQNxbM}

% Relevant video: ratio-test-example
\begin{question}
  Does the series \(\displaystyle\sum_{n=9}^\infty \left( \displaystyle\frac{2^{n}}{2^{2 \, n + 5} {\left(5 \, n + 1\right)}} \right)\) converge or diverge?
  
  \begin{solution}
    \begin{hint}
      This series is built out of powers and polynomials, so the ratio test might be helpful, since we can expect some things to cancel.
    \end{hint}
    \begin{hint}
      Let's set \(a_n = \displaystyle\frac{2^{n}}{2^{2 \, n + 5} {\left(5 \, n + 1\right)}}\).
    \end{hint}
    \begin{hint}
      Then \(a_{n+1} = \displaystyle\frac{2^{n + 1}}{2^{2 \, n + 7} {\left(5 \, n + 6\right)}}\).
    \end{hint}
    \begin{hint}
      So \(\displaystyle\displaystyle\frac{a_{n+1}}{a_n} = \displaystyle\frac{2^{2 \, n + 5} 2^{n + 1} {\left(5 \, n + 1\right)}}{2^{2 \, n + 7} 2^{n} {\left(5 \, n + 6\right)}}\).
    \end{hint}
    \begin{hint}
      Simplifying a bit yields \(\displaystyle\displaystyle\frac{a_{n+1}}{a_n} = \displaystyle\frac{5 \, n + 1}{2 \, {\left(5 \, n + 6\right)}}\).
    \end{hint}
    \begin{hint}
      In the ratio test, we compute \(L = \displaystyle\lim_{n \to \infty} \displaystyle\displaystyle\frac{a_{n+1}}{a_n}\).
    \end{hint}
    \begin{hint}
      In this case, \(L = \displaystyle\lim_{n \to \infty} \left( \displaystyle\frac{5 \, n + 1}{2 \, {\left(5 \, n + 6\right)}} \right) = \displaystyle\frac{1}{2}\).
    \end{hint}
    \begin{hint}
      Since \(L < 1\), we may conclude by the ratio test that the given series coverges.
    \end{hint}


    \begin{multiple-choice}
      \choice[correct]{The series converges.}
      \choice{The series diverges.}
    \end{multiple-choice}

  \end{solution}
\end{question}
            
\section{How do I apply the ratio test?}

Here is how to apply the ratio test.  Suppose \(a_n > 0\) and you want to determine whether the series \(\sum_{n=1}^\infty a_n\) converges or diverges.  All you need to do is compute \(L = \displaystyle\lim_{n\to\infty} a_{n+1}/a_n\), and if \(L < 1\), the series converges, while if \(L > 1\), the series diverges.  If \(L = 1\), then the test is inconclusive, and you'll have to try something else.

% Relevant video: ratio-test-statement
\begin{question}
  Does the series \(\displaystyle\sum_{n=6}^\infty \left( \displaystyle\frac{{\left(3 \, n + 2\right)} n!}{5^{n}} \right)\) converge or diverge?

  \begin{solution}
    \begin{hint}
      This series is built out of polynomials and powers and factorials, so the ratio test might be helpful, since we can expect a ton of things to cancel.
    \end{hint}
    \begin{hint}
      Let's set \(a_n = \displaystyle\frac{{\left(3 \, n + 2\right)} n!}{5^{n}}\).
    \end{hint}
    \begin{hint}
      Then \(a_{n+1} = \displaystyle\frac{{\left(3 \, n + 5\right)} \left(n + 1\right)!}{5^{n + 1}}\).
    \end{hint}
    \begin{hint}
      So \(\displaystyle\displaystyle\frac{a_{n+1}}{a_n} = \displaystyle\frac{5^{n} {\left(3 \, n + 5\right)} \left(n + 1\right)!}{5^{n + 1} {\left(3 \, n + 2\right)} n!}\).
    \end{hint}
    \begin{hint}
      The key to simplifying here is to note that \(\left(n + 1\right)! = n! \cdot (n + 1)\).
    \end{hint}
    \begin{hint}
      Simplifying a bit yields \(\displaystyle\displaystyle\frac{a_{n+1}}{a_n} = \displaystyle\frac{3 \, n^{2} + 8 \, n + 5}{5 \, {\left(3 \, n + 2\right)}}\).
    \end{hint}
    \begin{hint}
      In the ratio test, we compute \(L = \displaystyle\lim_{n \to \infty} \displaystyle\displaystyle\frac{a_{n+1}}{a_n}\).
    \end{hint}
    \begin{hint}
      In this case, \(L = \displaystyle\lim_{n \to \infty} \left( \displaystyle\frac{3 \, n^{2} + 8 \, n + 5}{5 \, {\left(3 \, n + 2\right)}} \right) = +\infty\).
    \end{hint}
    \begin{hint}
      Since \(L > 1\), we may conclude by the ratio test that the given series diverges.
      
    \end{hint}


    \begin{multiple-choice}
      \choice[correct]{The series diverges.}
      \choice{The series converges.}
    \end{multiple-choice}
    
  \end{solution}
\end{question}
            

\section{Does the ratio test always provide a useful answer?}

\begin{question}
  \begin{solution}
    \begin{multiple-choice}
      \choice[correct]{No.}
      \choice{Yes.}
    \end{multiple-choice}
  \end{solution}

  Indeed, the ratio test is not always helpful.  But it is not too
  hard to apply, so it is often worth trying!

  \youtube{https://www.youtube.com/watch?v=yRaG43ukXAI}
\end{question}

\section{What is the ratio test good for?}

The ratio test is powerful enough to address the convergence of \(\displaystyle\sum_{n=1}^\infty \frac{n!}{n^n}\) relatively easily.

\youtube{https://www.youtube.com/watch?v=g4_tcaiGjLc}

% Relevant video: ratio-test-one-over-e
\begin{question}
  Does the series \(\displaystyle\sum_{n=3}^\infty \left( \displaystyle\frac{10^{n} n!}{\left(3 \, n\right)^{n}} \right)\) converge or diverge?

  \begin{solution}
    \begin{hint}
      This series is built out of powers and factorials, so the ratio test might be helpful, since we can expect some cancellation.
    \end{hint}
    \begin{hint}
      Let's set \(a_n = \displaystyle\frac{10^{n} n!}{\left(3 \, n\right)^{n}}\).
    \end{hint}
    \begin{hint}
      Then \(a_{n+1} = \displaystyle\frac{10^{n + 1} \left(n + 1\right)!}{{\left(3 \, n + 3\right)}^{n + 1}}\).
    \end{hint}
    \begin{hint}
      So \(\displaystyle\displaystyle\frac{a_{n+1}}{a_n} = \displaystyle\frac{10^{n + 1} \left(3 \, n\right)^{n} \left(n + 1\right)!}{10^{n} {\left(3 \, n + 3\right)}^{n + 1} n!}\).
    \end{hint}
    \begin{hint}
      The key to simplifying here is to note that \(\left(n + 1\right)! = n! \cdot (n + 1)\).
    \end{hint}
    \begin{hint}
      Simplifying a bit yields \(\displaystyle\displaystyle\frac{a_{n+1}}{a_n} = \displaystyle\frac{10 \, \left(3 \, n\right)^{n}}{3 \, {\left(3 \, n + 3\right)}^{n}}\).
    \end{hint}
    \begin{hint}
      In the ratio test, we compute \(L = \displaystyle\lim_{n \to \infty} \displaystyle\displaystyle\frac{a_{n+1}}{a_n}\).
    \end{hint}
    \begin{hint}
      In this case, \(L = \displaystyle\lim_{n \to \infty} \left( \displaystyle\frac{10 \, \left(3 \, n\right)^{n}}{3 \, {\left(3 \, n + 3\right)}^{n}} \right)\).
    \end{hint}
    \begin{hint}
      One way to go about this is to use the fact that \(\displaystyle\lim_{n \to \infty} \left( \displaystyle\frac{n}{n + 1} \right)^{n} = e^{-1}\).
    \end{hint}
    \begin{hint}
      And so \(L = \displaystyle\lim_{n \to \infty} \left( \displaystyle\frac{10 \, \left(3 \, n\right)^{n}}{3 \, {\left(3 \, n + 3\right)}^{n}} \right) = \displaystyle\frac{10}{3} \, e^{\left(-1\right)}\).
    \end{hint}
    \begin{hint}
      Since \(e \approx 2.7183\), we find \(\displaystyle\frac{10}{3} \, e^{\left(-1\right)} \approx 1.2263\).
    \end{hint}
    \begin{hint}
      Since \(L > 1\), we may conclude by the ratio test that the given series diverges.
    \end{hint}

    
    \begin{multiple-choice}
      \choice[correct]{The series diverges.}
      \choice{The series converges.}
    \end{multiple-choice}
    
  \end{solution}
\end{question}

Incidentally, that is a fun example, since it leads us to think about how \(n!\) compares to \((n/e)^n\).

\youtube{https://www.youtube.com/watch?v=wnPBmJ70fzg}


\end{document}
